\documentclass[11pt]{article}
\usepackage{algorithm2e}
\usepackage[bottom]{footmisc} 
\usepackage[italian]{babel}
\usepackage[document]{ragged2e}
\justifying
\usepackage{amsfonts, amssymb, amsmath}
\usepackage{cancel}
\usepackage{float}
\usepackage{mathtools}
\usepackage[margin=3cm]{geometry}
% \setcounter{secnumdepth}{0}
\usepackage{hyperref}
\hypersetup{
    colorlinks,
    citecolor=black,
    filecolor=black,
    linkcolor=black,
    urlcolor=black
}
\usepackage{array}
\usepackage{makecell}
\usepackage{tikz}
\usetikzlibrary{arrows, positioning, quotes, shapes.geometric}
\usepackage{hyperref}
\usepackage[noabbrev,capitalize,italian]{cleveref}
\newcommand{\fref}[1]{\hyperref[#1]{\cref{#1}}}
\usepackage{makecell}
\usepackage{etoolbox}
\patchcmd{\thebibliography}{\section*{\refname}}{}{}{}

\tolerance=1
\emergencystretch=\maxdimen
\hyphenpenalty=10000
\hbadness=10000



% --- MACRO ---

% Box user story
% {US}{Punti}{DoD}{Test}{Epica}
\newcommand{\userstory}[5]{%
    \begin{center}
        \fbox{\parbox{15cm}{%
        \begin{minipage}{9.8cm}
            \textbf{US}: #1\\
            \textbf{Epica}: #5
        \end{minipage}
        \hfill
        \begin{minipage}{5cm}
            \begin{flushright}
                \textbf{Punti}: #2
            \end{flushright}
        \end{minipage}
        \vspace*{6pt}

        \textbf{DOD}: #3\\

        \def\temp{#4}\ifx\temp\empty
            %
        \else
            \textbf{Test}: #4\\
        \fi

    }}
    \end{center}
}



\begin{document}
\begin{titlepage}
    \begin{center}
        \vspace*{1.5cm}
            
        \Huge
        \textbf{Rapporto del TEAM 12}
            
        \vspace{0.3cm}
        \LARGE
        Tweet Analysis\\[0.2em]

        \vspace{1.5cm}
          
        \begin{minipage}[t]{0.47\textwidth}
            \begin{center}
                \parbox{50mm}{\centering\large {\bf Cheikh Ibrahim $\cdot$ Zaid} \\[0.2em] PO Operativo \\[0.3em] Matricola: \texttt{0000974909}}\\[2em]
                \parbox{50mm}{\centering\large {\bf Lee $\cdot$ Qun Hao Henry} \\[0.2em] Developer \\[0.3em] Matricola: \texttt{0000990259}}
            \end{center}
		\end{minipage}
		\hfill
		\begin{minipage}[t]{0.47\textwidth}\raggedleft
            \begin{center}
                \parbox{50mm}{\centering\large {\bf Xia $\cdot$ Tian Cheng} \\[0.2em] Scrum master \\[0.3em] Matricola: \texttt{0000975129}}\\[2em]
                \parbox{50mm}{\centering\large {\bf Paris $\cdot$ Manuel} \\[0.2em] Developer \\[0.3em] Matricola: \texttt{0000997526}}
            \end{center}
		\end{minipage}  
            
        \vspace{6cm}
            
        Anno accademico\\
        $2022 - 2023$
            
        \vspace{0.8cm}
            
            
        \Large
        Corso di Ingegneria del Software\\
        Alma Mater Studiorum $\cdot$ Università di Bologna\\
            
    \end{center}
\end{titlepage}
\pagebreak


\tableofcontents
\newpage


\section{Descrizione del prodotto}
Il prodotto da realizzare è un client Twitter in grado di fornire analisi e funzionalità specifiche per determinati argomenti.

\subsection{Scope}
Il backlog è stato partizionato in cinque epiche:
\begin{center}
    \begin{tabular}{ | m{8em} | m{11cm} | }
        \hline
        {\textbf{Epica}} & {\textbf{Descrizione}} \\
        \hline
        Raccolta e analisi di tweet & Prime funzionalità per ricercare e analizzare tweet \\ 
        \hline
        Scacchi & Partita a scacchi contro gli utenti di Twitter che scelgono a maggioranza la mossa \\ 
        \hline
        Reazione a catena & Raccolta dei tentativi degli utenti di Twitter di indovinare la \textit{catena finale} \\ 
        \hline
        L'Eredità & Raccolta dei tentativi degli utenti di Twitter di indovinare la \textit{ghigliottina} \\ 
        \hline
        Fantacitorio & Raccolta e visualizzazione dei punteggi del Fantacitorio e ricerca delle squadre degli altri utenti \\ 
        \hline
    \end{tabular}
\end{center}
Le user stories contenute in ciascuna epica sono descritte nella \fref{sprint_description}.

\subsection{Casi d'uso}
\begin{figure}[H]
    \centering
    \includegraphics[width=14cm]{./img/usecase/tweet.png}
    \caption{Casi d'uso per l'epica: Raccolta e analisi di tweet}
\end{figure}

\begin{figure}[H]
    \centering
    \includegraphics[width=10cm]{./img/usecase/tvgames.png}
    \caption{Casi d'uso per l'epica: Giochi televisivi}
\end{figure}

\begin{figure}[H]
    \centering
    \includegraphics[width=10cm]{./img/usecase/chess.png}
    \caption{Casi d'uso per l'epica: Scacchi}
\end{figure}

\begin{figure}[H]
    \centering
    \includegraphics[width=11cm]{./img/usecase/fantacitorio.png}
    \caption{Casi d'uso per l'epica: Fantacitorio}
\end{figure}


\subsection{Diagramma delle classi}


%%%

\newpage
\section{Descrizione degli sprint} \label{sprint_description}
Sono stati svolti quattro sprint della durata di 14 giorni ciascuno.\\
La stima dei punti delle user stories è stata effettuata con una scala da 0 a 10, valutando separatamente il frontend dal backend. 
Il punteggio complessivo è quindi ottenuto dalla somma di quest'ultimi.

\subsection{Sprint 1}
\subsubsection{Sprint goal}
Lo sprint è stato principalmente dedicato a studiare le API di Twitter e produrre le prime funzionalità per la visualizzazione e l'analisi dei tweet.\\
In particolare le feature pianificate per lo sprint sono state:
\begin{itemize}
    \item Ricerca di tweet per username
    \item Ricerca di tweet per hashtag
    \item Analisi dei tweet tramite componenti grafiche (grafico a torta per il sentiment analysis, grafico a barre per la frequenza dei tweet e word cloud)
\end{itemize}


\subsubsection{Backlog}
\userstory%
{Come utente interessato ai tweet,\\voglio poter cercare dei tweet per hashtag\\per leggerli.}%
{8\\(3 frontend + 5 backend)}%
{L'utente, cercando un hashtag in un apposito textbox, è in grado di leggere tutti i tweet correlati visualizzando:
nome account Twitter, username, immagine profilo, contenuto Tweet (testo + foto e video), data e ora, luogo (se applicabile), numero like, numero commenti, numero retweet}%
{Richiamare l'API implementata, verificare che il formato sia corretto e che il contenuti dei tweet contenga l'hashtag ricercato}

\userstory%
{Come utente interessato ai tweet,\\voglio poter cercare dei tweet per nome utente\\per leggerli.}%
{8\\(3 frontend + 5 backend)}%
{L'utente, cercando un nome utente in un apposito textbox, è in grado di leggere tutti i tweet correlati visualizzando:
nome account Twitter, username, immagine profilo, contenuto Tweet (testo + foto e video), data e ora, luogo (se applicabile), numero like, numero commenti, numero retweet}%
{Richiamare l'API implementata, verificare che il formato sia corretto e che l'autore dei tweet sia quello ricercato}

\userstory%
{Come analista,\\voglio poter analizzare il sentimento\\per stabilire se un tweet è positivo o meno.}%
{9\\(2 frontend + 7 backend)}%
{L'utente, dato un tweet, vede se è positivo, negativo o neutro tramite immagine o testo.}%
{Analizzare frasi di cui è noto il sentimento}

\userstory%
{Come analista,\\voglio vedere un grafico a barre\\per vedere il numero di tweet nell'unità di tempo.}%
{2\\(2 frontend)}%
{L'utente apre una pagina web contenente il grafico a barre con il numero di tweet nell'unità di tempo.}%
{Manualmente verificare che il grafico sia corretto}

\userstory%
{Come analista,\\voglio vedere un grafico a torta\\per vedere il rapporto di sentiment positivi, negativi e neutri.}%
{2\\(2 frontend)}%
{L'utente apre una pagina web contenente il grafico a torta con sentiment positivi, negativi e neutri.}%
{Manualmente verificare che il grafico sia corretto}

\userstory%
{Come analista,\\voglio vedere una term cloud\\per vedere le parole più utilizzate nei tweet.}%
{4\\(2 frontend + 2 backend)}%
{L'utente apre una pagina web contenente una term cloud con le parole più utilizzate.}%
{Manualmente verificare che il grafico sia corretto}


\subsubsection{Burndown}
\begin{figure}[H]
    \centering
    \includegraphics[width=15cm]{./img/sprint1/burndown.png}
    \caption{Burndown}
\end{figure}
\begin{figure}[H]
    \centering
    \includegraphics[width=15cm]{./img/sprint1/worktime.png}
    \caption{Progresso dei punti (asse a sinistra) e ore di lavoro (asse a destra)}
\end{figure}


\subsubsection{Retrospettiva}
\begin{figure}[H]
    \centering
    \includegraphics[width=15cm]{./img/sprint1/preretrospettiva.png}
    \caption{Pre-retrospettiva del 27/10/2022}
\end{figure}

\begin{figure}[H]
    \centering
    \includegraphics[width=15cm]{./img/sprint1/retrospettiva.png}
    \caption{Retrospettiva del 01/11/2022}
\end{figure}
\newpage

\subsection{Sprint 2}
\subsubsection{Sprint goal}
L'obiettivo dello sprint è stato quello di concludere l'epica riguardante la visualizzazione e l'analisi dei tweet.\\
Nello specifico, sono state implementate le seguenti funzionalità:
\begin{itemize}
    \item Ricerca di tweet per intervallo temporale (richiesto dal cliente allo sprint review precedente)
    \item Ricerca di un determinato numero di tweet con una singola ricerca (richiesto dal cliente allo sprint review precedente)
    \item Ricerca di tweet per parola chiave
    \item Mappa per visualizzare la posizione dei tweet con geolocalizzazione
    \item Raccolta di tweet in tempo reale
\end{itemize}


\subsubsection{Backlog}
\userstory%
{Come utente interessato a vedere tweet,\\voglio scegliere un numero di tweet da poter caricare in una volta\\per poterli analizzare in modo aggregato.}%
{5\\(3 frontend + 2 backend)}%
{Possibilità di ricercare un largo numero di tweet scrivendone la quantità in un textbox.\\
Tale quantità serve per la ricerca iniziale e anche per mostrare pagine successive. 
È possibile inoltre ricercare un numero di tweet inizialmente per poi cambiare quantità e cercare una pagina successiva 
(esempio: ricerca di 150 tweet iniziali, poi viene modificata la quantità (dallo stesso textbox) in 20 e si preme su “Pagina successiva”. Il numero di tweet così mostrato diventa 170).}%
{Verificare che il numero di tweet raccolto corrisponda con quello richiesto.}

\userstory%
{Come utente interessato a vedere tweet,\\voglio scegliere un intervallo di tempo in cui raccogliere tweet\\per analizzarne le tendenze storiche.}%
{5\\(3 frontend + 2 backend)}%
{Possibilità di ricercare dei tweet dato un intervallo temporale. Non deve essere possibile cercare tweet nel futuro.\\
Non deve essere possibile cliccare su “Prossima pagina” quando non ci sono più tweet da visualizzare.}%
{Verificare che i tweet raccolti siano compresi nell'intervallo temporale.}

\userstory%
{Come utente interessato a vedere tweet,\\voglio poter cercare dei tweet per parola chiave\\per vedere cosa ne pensa la gente a riguardo.}%
{4\\(2 frontend + 2 backend)}%
{Possibilità di cercare tweet per parola o frase chiave. I grafici già presenti devono funzionare anche con questa ricerca.}%
{Richiamare l'API implementata, verificare che il formato sia corretto e che il contenuti dei tweet contenga la parola chiave ricercata.}

\userstory%
{Come utente interessato ai tweet,\\voglio poter visualizzare su una mappa la posizione dei tweet cercati\\per avere un'idea della località dalla quale sono stati pubblicati.}%
{7\\(3 frontend + 4 backend)}%
{Possibilità di visualizzare una mappa con le posizioni dei tweet ricercati.\\
Se sono presenti più tweet nella stessa zona è possibile aggregarli (in base alla distanza) e mostrare un unico valore, 
ovvero il numero di tweet in tale zona. La mappa deve essere sempre visibile anche durante lo scorrimento della pagina (come i grafici).}%
{Manualmente verificare che la mappa contenga i marker quando sono presenti tweet con geolocalizzazione.}

\userstory%
{Come utente,\\voglio vedere la posizione di tutti i tweet di una data persona\\per conoscere i suoi spostamenti.}%
{5\\(5 frontend + 0 backend)}%
{Possibilità di inserire il nome utente di una persona e visualizzare su una mappa le posizioni dei suoi tweet e i suoi spostamenti.\\
Per gli spostamenti si mostrano sulla mappa delle frecce basandosi sulla posizione e sulla data del tweet.}%
{Manualmente verificare che la mappa contenga i marker quando sono presenti tweet con geolocalizzazione.}

\userstory%
{Come lettore di tweet,\\voglio poter vedere i tweet che ricerco in tempo reale\\per sapere cosa la gente posta.}%
{10\\(3 frontend + 7 backend)}%
{Poter scrivere una ricerca e, al click di un pulsante “Live”, vedere tutti i tweet pubblicati in tempo reale a partire da quel momento.}%
{Verificare che il socket implementato restituisca i tweet ricercati, quando disponibili.}

\newpage
\subsubsection{Esito sprint}
Lo sprint è terminato con la conclusione di tutte le user stories pianificate.\\
Il lavoro si è svolto in linea con l'andamento ideale dei punti. 
Si è osservato una leggera sovrastima delle user stories che ha portato alla conclusione anticipata (di un giorno) dello sprint.\\
\begin{figure}[H]
    \centering
    \includegraphics[width=15cm]{./img/sprint2/burndown.png}
    \caption{Burndown sprint 2}
\end{figure}
Le ore di lavoro rispettano il monte ore e sono state distribute principalmente in due periodi di maggiore produttività.
\begin{figure}[H]
    \centering
    \includegraphics[width=15cm]{./img/sprint2/worktime.png}
    \caption{Progresso dei punti (asse a sinistra) e ore di lavoro (asse a destra)}
\end{figure}


\subsubsection{Sprint review}
Alla sprint review non sono emerse nuove richieste da parte del cliente.


\newpage
\subsubsection{Retrospettiva}
\subsubsection*{Pre-retrospettiva}
Alla pre-retrospettiva effettuata a metà sprint, sono emerse le seguenti problematiche:
\begin{itemize}
    \item Mancanza di test adeguati per il frontend
    \item Le user stories sono state sovrastimate
\end{itemize}
\begin{figure}[H]
    \centering
    \includegraphics[width=15cm]{./img/sprint2/preretrospettiva.png}
    \caption{Pre-retrospettiva del 10/11/2022}
\end{figure}

\subsection*{Retrospettiva}
Alla retrospettiva di fine sprint sono state confermate le problematiche della pre-retrospettiva.
\newpage

\subsection{Sprint 3}
\subsubsection{Sprint goal}
Per lo sprint è stata pianificata l'implementazione delle epiche riguardanti i giochi televisivi (\textit{L'Eredità} e \textit{Reazione a Catena}) e
l'inizio dell'epica riguardante gli scacchi.\\
In particolare sono state previste le seguenti funzionalità:
\begin{itemize}
    \item Visualizzare gli utenti che tentano di indovinare la parola del giorno
    \item Visualizzare la parola vincente
    \item Visualizzare gli utenti che indovinano la parola
    \item Visualizzare la posizione degli utenti che tentano di indovinare
    \item Possibilità di muovere le proprie pedine su una scacchiera
\end{itemize}


\subsubsection{Backlog}
\userstory%
{Come spettatore de \#leredita,\\voglio raccogliere i tweet di chi prova a indovinare la \textit{Ghigliottina},\\per visualizzare, in ordine temporale, tutti coloro che provano ad indovinare.}%
{5\\(2 frontend + 3 backend)}%
{Possibilità di visualizzare una pagina con tutti i tweet di tutte le persone che usano \#leredita e provano ad indovinare la \textit{Ghigliottina}.}%
{Cercando per data, verificare che vengano visualizzati i tentativi del giorno.}
{L'Eredità}

\userstory%
{Come spettatore de \#leredita,\\voglio visualizzare, su una mappa, la posizione di tutti coloro che provano ad indovinare\\per conoscere la posizione dei giocatori.}%
{4\\(4 frontend)}%
{Possibilità di visualizzare una pagina con tutte le posizioni su una mappa dei tweet di tutte le persone che usano \#leredita e provano ad indovinare la \textit{Ghigliottina}.}%
{Manualmente verificare che nella mappa siano presenti i marker dei tweet con la geolocalizzazione.}
{L'Eredità}

\userstory%
{Come spettatore de \#leredita,\\voglio visualizzare tutti coloro che indovinano la \textit{Ghigliottina}\\per sapere chi ha indovinato.}%
{7\\(4 frontend + 3 backend)}%
{Possibilità di visualizzare la parola del giorno assieme a tutti i vincitori che hanno indovinato.}%
{Cercando per data, verificare che venga trovata la parola del giorno.}
{L'Eredità}

\userstory%
{Come spettatore di \#reazioneacatena,\\voglio raccogliere i tweet di chi prova a indovinare l'ultima parola,\\per visualizzare, in ordine temporale, tutti coloro che provano ad indovinare.}%
{2\\(2 backend)}%
{Possibilità di visualizzare una pagina con tutti i tweet di tutte le persone che usano \#reazioneacatena e provano ad indovinare la parola finale.}%
{Cercando per data, verificare che vengano visualizzati i tentativi del giorno.}
{Reazione a catena}

\userstory%
{Come spettatore di \#reazioneacatena,\\voglio visualizzare, su una mappa, la posizione di tutti coloro che provano ad indovinare\\per conoscere la posizione di giocatori.}%
{0\\Segue dalla US de \#leredita}%
{Possibilità di visualizzare una pagina con tutte le posizioni su una mappa di tweet di tutte le persone che usano \#reazioneacatena e provano ad indovinare la parola finale.}%
{Manualmente verificare che nella mappa siano presenti i marker dei tweet con la geolocalizzazione.}
{Reazione a catena}

\userstory%
{Come spettatore de \#reazioneacatena,\\voglio visualizzare tutti coloro che indovinano l'ultima parola\\per sapere chi ha indovinato.}%
{2\\(2 backend)}%
{Possibilità di visualizzare la parola del giorno assieme a tutti i vincitori che hanno indovinato.}%
{Cercando per data, verificare che venga trovata la parola del giorno.}
{Reazione a catena}

\userstory%
{Come giocatore di scacchi,\\voglio poter muovere una pedina\\per fare la mia mossa.}%
{12\\(6 frontend + 6 backend)}%
{Possibilità di fare una mossa a scacchi e visualizzarla.}%
{Provare a effettuare mosse valide e invalide.}
{Scacchi}


\newpage
\subsubsection{Esito sprint}
Lo sprint è terminato con la conclusione di tutte le user stories pianificate.\\
\begin{figure}[H]
    \centering
    \includegraphics[width=15cm]{./img/sprint3/burndown.png}
    \caption{Burndown sprint 3}
\end{figure}
Il lavoro è stato però distribuito in maniera disomogenea, infatti, nella prima metà dello sprint non sono stati fatti progressi significativi,
mentre tutto il valore è stato portato nella seconda metà dello sprint. 
Conseguentemente anche le ore di lavoro sono tutte distribuite nel secondo periodo dello sprint.\\
Il monte ore obiettivo non è stato raggiunto, ma ciò era atteso in quanto durante lo sprint planning era già stata prevista una minore quantità di lavoro.
\begin{figure}[H]
    \centering
    \includegraphics[width=15cm]{./img/sprint3/worktime.png}
    \caption{Progresso dei punti (asse a sinistra) e ore di lavoro (asse a destra)}
\end{figure}


\subsubsection{Sprint review}
Alla sprint review il cliente ha espresso le seguenti richieste:
\begin{itemize}
    \item Revisione del grafico a barre della frequenza dei tweet in situazioni in cui sono presenti solo tweet di una giornata
    \item Aggiungere ai giochi televisivi la visualizzazione dei giocatori più vincenti in un periodo di tempo
\end{itemize} 
Inoltre è stata aggiunta alle specifiche, con priorità massima, l'implementazione di funzionalità riguardanti il Fantacitorio.


\newpage
\subsubsection{Retrospettiva}
Dalla retrospettiva sono emerse le seguenti problematiche:
\begin{itemize}
    \item Abbiamo sottovalutato lo sprint e lavorato poco di conseguenza
    \item La DoD di alcune user stories risultavano ambigue
\end{itemize}
\begin{figure}[H]
    \centering
    \includegraphics[width=15cm]{./img/sprint3/retrospettiva.png}
    \caption{Retrospettiva del 02/12/2022}
\end{figure}

\newpage

\subsection{Sprint 4}
\subsubsection{Sprint goal}
\subsubsection{Backlog}
\subsubsection{Burndown}
\begin{figure}[H]
    \centering
    \includegraphics[width=15cm]{./img/sprint4/burndown.png}
    \caption{Burndown}
\end{figure}
\begin{figure}[H]
    \centering
    \includegraphics[width=15cm]{./img/sprint4/worktime.png}
    \caption{Progresso dei punti (asse a sinistra) e ore di lavoro (asse a destra)}
\end{figure}
\subsubsection{Retrospettiva}
\begin{figure}[H]
    \centering
    \includegraphics[width=15cm]{./img/sprint4/retrospettiva.png}
    \caption{Pre-retrospettiva del 02/12/2022}
\end{figure}
\newpage

%%%

\newpage
\section{Descrizione del processo}

\subsection{Flusso di lavoro}
Le responsabilità di ciascun componente del gruppo nel processo di sviluppo sono stati i 
seguenti\footnote{Ad eccezione dello sprint 3 durante il quale è stato sperimentato l'inversione dei ruoli del team di sviluppo}:
\begin{center}
    \begin{tabular}{ | m{4cm} | m{10cm} | }
        \hline
        {\textbf{Membro}} & {\textbf{Responsabilità}} \\
        \hline
        \makecell[cl]{Cheikh Ibrahim Zaid\\{\footnotesize PO Operativo}} & Sviluppo backend. Testing e quality assurance. \\ 
        \hline
        \makecell[cl]{Lee Qun Hao Henry\\{\footnotesize Developer}} & Sviluppo frontend. \\ 
        \hline
        \makecell[cl]{Paris Manuel\\{\footnotesize Developer}} & Sviluppo backend. \\ 
        \hline
        \makecell[cl]{Xia Tian Cheng\\{\footnotesize Scrum master}} & Sviluppo fullstack. Sistemista e CI/CD. \\ 
        \hline
    \end{tabular}
\end{center}

\subsubsection{Git}
L'organizzazione del repository git segue la metodologia \textit{gitflow} con un branch principale (\texttt{main}) che dopo ogni sprint viene allineato
con il branch di sviluppo (\texttt{dev}).\\
Più nello specifico, il flusso del lavoro ha seguito il seguente schema:
\begin{center}
\begin{tikzpicture}[auto, thick,
        node distance = 1cm and 1cm,
        pipestep/.style = {draw, align=center, minimum width=2cm, minimum height=1cm, rounded corners},
    ]
    \node[pipestep]                                     (taiga) {Selezione\\task su Taiga};
    \node[pipestep, right=of taiga]                     (git_branch) {Sviluppo su\\branch separato};
    \node[pipestep, right=of git_branch]                (mr) {Creazione\\merge request};
    \node[pipestep, below=of mr]                        (mr_approval1) {Approvazione\\tecnica};
    \node[pipestep, right=of mr_approval1]              (mr_approval2) {Approvazione\\del PO};
    \coordinate[right=of mr, xshift=0.8cm]              (padding);
    \node[pipestep, right=of padding]                   (merge) {Merge nel\\branch \texttt{dev}};
    \node[pipestep, below=of merge, yshift=-2cm]        (prod) {Merge nel\\branch \texttt{main}};
    %
    \draw[-stealth']  (taiga) -- (git_branch) node[midway, anchor=east] {};
    \draw[-stealth']  (git_branch) -- (mr) node[midway, anchor=east] {};
    \draw[-stealth']  (mr) -- (mr_approval1) node[midway, anchor=east] {};
    \draw[-stealth']  (mr_approval1) -- (mr_approval2) node[midway, anchor=east] {};
    \draw[-stealth']  (mr_approval2) -- (merge) node[midway, anchor=east] {};
    \draw[dashed,->]  (merge) -- (prod) node[midway, anchor=west, text width=1.5cm] {Termine sprint};
    \draw[-stealth']  (merge.90) arc (-30:205:4mm) node[midway, anchor=west, text width=2cm, xshift=-2.5cm] {Risoluzione code smell};
    % 
\end{tikzpicture}
\end{center}

\subsubsection{Bug tracking}
Per il tracciamento di problemi e bug sono stati usati gli strumenti disponibili in Gitlab e Taiga.
In particolare è stato utilizzato la funzionalità \textit{Issues} di Gitlab per creare e segnalare bug. 
Contemporaneamente è stato impostato Taiga in modo tale che venga sincronizzato con Gitlab.\\
In questo modo sono visibile su entrambi gli strumenti le segnalazioni e il loro stato, permettendo una migliore pianificazione in fase di sprint planning e
una maggiore organizzazione durante lo sprint.


\subsection{Team building}
\subsubsection{Scrumble}
La partita a Scrumble è stata giocata allo sprint 0, durante la fase preparatoria del progetto.\\
Lo scopo della sessione di team building era quella di prendere maggiore confidenza con la metodologia Scrum e 
confermare i ruoli all'interno del gruppo (Autovalutazione \cite{gqm_scrumble}).

\subsubsection{Escape the Boom}
La partita a Scrumble è stata giocata allo sprint 1.\\
L'obiettivo era quello di rafforzare la comunicazione nel team, che, come segnalato dal team alla retrospettiva (\fref{sprint1_retrospettiva}),
era carente e poco significativa (Autovalutazione \cite{gqm_escapetheboom}).


\subsection{Gitinspector}



\subsection{Deployment}
Il seguente schema rappresenta le fasi della pipeline definita per il progetto:
\begin{center}
\begin{tikzpicture}[auto, thick,
        node distance = 1cm and 1 cm,
        pipestep/.style = {draw, align=center, minimum width=2.5cm, minimum height=1cm, rounded corners},
    ]
    \node[pipestep]                                     (gitlab) {\includegraphics[width=1cm]{./img/icons/firefox.png}\\\textit{Push} su Gitlab};
    \node[pipestep, below=of gitlab]                    (jenkins) {\includegraphics[width=0.9cm]{./img/icons/jenkins.png}\\Avvio pipeline\\Jenkins};
    \node[pipestep, right=of jenkins]                   (testing) {\includegraphics[width=1cm]{./img/icons/jest.png}\\Testing};
    \coordinate[right=of testing, xshift=3cm]           (padding);
    \node[pipestep, above=of padding, yshift=0.5cm]     (deploy_prod) {\includegraphics[width=1cm]{./img/icons/docker.png}\\Deploy produzione};
    \node[pipestep, below=of padding, yshift=-0.5cm]    (sonarqube) {\includegraphics[width=1cm]{./img/icons/sonarqube.png}\\Scansione\\Sonarqube};
    \node[pipestep, right=of sonarqube]                 (deploy_dev) {\includegraphics[width=1cm]{./img/icons/docker.png}\\Deploy\\ambiente di test};
    \node[pipestep, below=of testing]                   (mattermost) {\includegraphics[width=1cm]{./img/icons/mattermost.png}\\Segnalazione su\\Mattermost};
    %
    \draw[-stealth']  (gitlab) -- (jenkins) node[midway, anchor=east, xshift=0.9cm, yshift=0.1cm, fill=white] {Webhook};
    \draw[-stealth']  (jenkins) -- (testing) node[midway, anchor=east] {};
    \draw[-stealth']  (testing.east) -- (deploy_prod.west) node[midway, anchor=east, xshift=1.6cm, yshift=-0.1cm, fill=white] {Branch \texttt{main}};
    \draw[-stealth']  (testing.east) -- (sonarqube.west) node[midway, anchor=east, xshift=1.2cm, yshift=0.1cm, fill=white] {Branch \texttt{dev}};
    \draw[-stealth']  (sonarqube) -- (deploy_dev) node[midway] {};
    \draw[-stealth']  (testing) -- (mattermost) node[midway, anchor=east, xshift=1.1cm, yshift=0.1cm, fill=white] {Fallimento};
    % 
\end{tikzpicture}
\end{center}
Sonarqube è definito solo sul branch di sviluppo (\texttt{dev}) in quanto la versione \textit{community} è limitata alla scansione di un unico branch.\\
Il deploy avviene su container Docker allineati al codice del branch di riferimento. Il database risiede su un container separato.


\newpage
\section{Artefatti}
\bibliographystyle{unsrt}
\bibliography{references}

\end{document}

