\documentclass[11pt]{article}
\usepackage{algorithm2e}
\usepackage[italian]{babel}
\usepackage[document]{ragged2e}
\justifying
\usepackage{amsfonts, amssymb, amsmath}
\usepackage{cancel}
\usepackage{float}
\usepackage{mathtools}
\usepackage[margin=3cm]{geometry}
% \setcounter{secnumdepth}{0}
\usepackage{hyperref}
\hypersetup{
    colorlinks,
    citecolor=black,
    filecolor=black,
    linkcolor=black,
    urlcolor=black
}
\usepackage{array}
\usepackage{makecell}

\tolerance=1
\emergencystretch=\maxdimen
\hyphenpenalty=10000
\hbadness=10000



% --- MACRO ---

% Box user story
\newcommand{\userstory}[4]{%
    \begin{center}
        \fbox{\parbox{15cm}{%
        \begin{minipage}{9.8cm}
            \textbf{US}: #1
        \end{minipage}
        \hfill
        \begin{minipage}{5cm}
            \begin{flushright}
                \textbf{Punti}: #2
            \end{flushright}
        \end{minipage}
        \vspace*{6pt}

        \textbf{DOD}: #3\\

        \def\temp{#4}\ifx\temp\empty
            %
        \else
            \textbf{Test}: #4\\
        \fi

    }}
    \end{center}
}



\begin{document}
\begin{titlepage}
    \begin{center}
        \vspace*{1.5cm}
            
        \Huge
        \textbf{Tweet Analysis}
            
        \vspace{0.3cm}
        \LARGE
        Relazione finale\\[0.2em]

        \vspace{1.5cm}
          
        \begin{minipage}[t]{0.47\textwidth}
            \begin{center}
                \parbox{50mm}{\centering\large {\bf Cheikh Ibrahim $\cdot$ Zaid} \\[0.2em] PO Operativo \\[0.3em] Matricola: \texttt{0000974909}}\\[2em]
                \parbox{50mm}{\centering\large {\bf Lee $\cdot$ Qun Hao Henry} \\[0.2em] Developer \\[0.3em] Matricola: \texttt{0000990259}}
            \end{center}
		\end{minipage}
		\hfill
		\begin{minipage}[t]{0.47\textwidth}\raggedleft
            \begin{center}
                \parbox{50mm}{\centering\large {\bf Xia $\cdot$ Tian Cheng} \\[0.2em] Scrum master \\[0.3em] Matricola: \texttt{0000975129}}\\[2em]
                \parbox{50mm}{\centering\large {\bf Paris $\cdot$ Manuel} \\[0.2em] Developer \\[0.3em] Matricola: \texttt{0000997526}}
            \end{center}
		\end{minipage}  
            
        \vspace{6cm}
            
        Anno accademico\\
        $2022 - 2023$
            
        \vspace{0.8cm}
            
            
        \Large
        Corso di Ingegneria del Software\\
        Alma Mater Studiorum $\cdot$ Università di Bologna\\
            
    \end{center}
\end{titlepage}
\pagebreak


\tableofcontents
\newpage


\section{Descrizione del prodotto}

\subsection{Scope}

\subsection{Casi d'uso}

\subsection{Diagramma delle classi}

%%%

\newpage
\section{Descrizione degli sprint}
Sono stati svolti quattro sprint della durata di 14 giorni ciascuno.\\
La stima dei punti delle user stories è stata effettuata con una scala da 0 a 10, valutando separatamente il frontend dal backend. 
Il punteggio complessivo è quindi ottenuto dalla somma di quest'ultimi.

\subsection{Sprint 1}
\subsubsection{Sprint goal}
Lo sprint è stato principalmente dedicato a studiare le API di Twitter e produrre le prime funzionalità per la visualizzazione e l'analisi dei tweet.\\
In particolare le feature pianificate per lo sprint sono state:
\begin{itemize}
    \item Ricerca di tweet per username
    \item Ricerca di tweet per hashtag
    \item Analisi dei tweet tramite componenti grafiche (grafico a torta per il sentiment analysis, grafico a barre per la frequenza dei tweet e word cloud)
\end{itemize}


\subsubsection{Backlog}
\userstory%
{Come utente interessato ai tweet,\\voglio poter cercare dei tweet per hashtag\\per leggerli.}%
{8\\(3 frontend + 5 backend)}%
{L'utente, cercando un hashtag in un apposito textbox, è in grado di leggere tutti i tweet correlati visualizzando:
nome account Twitter, username, immagine profilo, contenuto Tweet (testo + foto e video), data e ora, luogo (se applicabile), numero like, numero commenti, numero retweet}%
{Richiamare l'API implementata, verificare che il formato sia corretto e che il contenuti dei tweet contenga l'hashtag ricercato}

\userstory%
{Come utente interessato ai tweet,\\voglio poter cercare dei tweet per nome utente\\per leggerli.}%
{8\\(3 frontend + 5 backend)}%
{L'utente, cercando un nome utente in un apposito textbox, è in grado di leggere tutti i tweet correlati visualizzando:
nome account Twitter, username, immagine profilo, contenuto Tweet (testo + foto e video), data e ora, luogo (se applicabile), numero like, numero commenti, numero retweet}%
{Richiamare l'API implementata, verificare che il formato sia corretto e che l'autore dei tweet sia quello ricercato}

\userstory%
{Come analista,\\voglio poter analizzare il sentimento\\per stabilire se un tweet è positivo o meno.}%
{9\\(2 frontend + 7 backend)}%
{L'utente, dato un tweet, vede se è positivo, negativo o neutro tramite immagine o testo.}%
{Analizzare frasi di cui è noto il sentimento}

\userstory%
{Come analista,\\voglio vedere un grafico a barre\\per vedere il numero di tweet nell'unità di tempo.}%
{2\\(2 frontend)}%
{L'utente apre una pagina web contenente il grafico a barre con il numero di tweet nell'unità di tempo.}%
{Manualmente verificare che il grafico sia corretto}

\userstory%
{Come analista,\\voglio vedere un grafico a torta\\per vedere il rapporto di sentiment positivi, negativi e neutri.}%
{2\\(2 frontend)}%
{L'utente apre una pagina web contenente il grafico a torta con sentiment positivi, negativi e neutri.}%
{Manualmente verificare che il grafico sia corretto}

\userstory%
{Come analista,\\voglio vedere una term cloud\\per vedere le parole più utilizzate nei tweet.}%
{4\\(2 frontend + 2 backend)}%
{L'utente apre una pagina web contenente una term cloud con le parole più utilizzate.}%
{Manualmente verificare che il grafico sia corretto}


\subsubsection{Burndown}
\begin{figure}[H]
    \centering
    \includegraphics[width=15cm]{./img/sprint1/burndown.png}
    \caption{Burndown}
\end{figure}
\begin{figure}[H]
    \centering
    \includegraphics[width=15cm]{./img/sprint1/worktime.png}
    \caption{Progresso dei punti (asse a sinistra) e ore di lavoro (asse a destra)}
\end{figure}

\subsubsection{Retrospettiva}
\begin{figure}[H]
    \centering
    \includegraphics[width=15cm]{./img/sprint1/preretrospettiva.png}
    \caption{Pre-retrospettiva del 27/10/2022}
\end{figure}

\begin{figure}[H]
    \centering
    \includegraphics[width=15cm]{./img/sprint1/retrospettiva.png}
    \caption{Retrospettiva del 01/11/2022}
\end{figure}


\subsection{Sprint 2}
\subsubsection{Sprint goal}

\subsubsection{Backlog}

\subsubsection{Burndown}
\begin{figure}[H]
    \centering
    \includegraphics[width=15cm]{./img/sprint2/burndown.png}
    \caption{Burndown}
\end{figure}
\begin{figure}[H]
    \centering
    \includegraphics[width=15cm]{./img/sprint2/worktime.png}
    \caption{Progresso dei punti (asse a sinistra) e ore di lavoro (asse a destra)}
\end{figure}

\subsubsection{Retrospettiva}
\begin{figure}[H]
    \centering
    \includegraphics[width=15cm]{./img/sprint2/preretrospettiva.png}
    \caption{Pre-retrospettiva del 10/11/2022}
\end{figure}


\subsection{Sprint 3}
\subsubsection{Sprint goal}

\subsubsection{Backlog}

\subsubsection{Burndown}
\begin{figure}[H]
    \centering
    \includegraphics[width=15cm]{./img/sprint3/burndown.png}
    \caption{Burndown}
\end{figure}
\begin{figure}[H]
    \centering
    \includegraphics[width=15cm]{./img/sprint3/worktime.png}
    \caption{Progresso dei punti (asse a sinistra) e ore di lavoro (asse a destra)}
\end{figure}

\subsubsection{Retrospettiva}
\begin{figure}[H]
    \centering
    \includegraphics[width=15cm]{./img/sprint3/retrospettiva.png}
    \caption{Pre-retrospettiva del 02/12/2022}
\end{figure}


\subsection{Sprint 4}
\subsubsection{Sprint goal}
\subsubsection{Backlog}
\subsubsection{Burndown}
\subsubsection{Retrospettiva}

%%%

\newpage
\section{Descrizione del processo}

\subsection{Team building}
\subsubsection{Scrumble}
\subsubsection{Escape the Boom}

\subsection{Gitinspector}

\subsection{Retrospettiva finale}

\subsection{Deployment}


\newpage
\section{Artefatti}

\end{document}

