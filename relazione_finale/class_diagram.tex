Di seguito sono descritti, sottoforma di schede CRC, gli oggetti e le funzioni suddivisi per epica del backlog.\\
Per molte funzionalità del prodotto, si è ritenuto più adatto seguire un approccio funzionale (dividendo in moduli) invece di utilizzare un approccio a oggetti.\\
Nonostante ciò, si è preferito descrivere la progettazione come schede CRC, anche se le entità non sono propriamente classi,
in quanto per ciascun modulo è possibile individuare le responsabilità e le dipendenze.

\subsubsection*{Raccolta e analisi di tweet}
%%% Raccolta
\crc{fetch/keyword}{%
    \item Raccolta di tweet per parola chiave
}{%
}

\crc{fetch/user}{%
    \item Raccolta di tweet per nome utente
}{%
}

\crc{fetch/multiple\_tweets}{%
    \item Raccolta di un determinato numero (potenzialmente alto) di tweet
}{%
    \item \texttt{fetch/keyword}
    \item \texttt{fetch/user}
}

\crc{fetch/stream}{%
    \item Apertura di uno stream per la raccolta di tweet in tempo reale
    \item Lettura dallo stream
    \item Chiusura dello stream
}{%
}

%%% Analisi
\crc{analysis/language}{%
    \item Rilevazione della lingua di un testo
}{%
}

\crc{analysis/sentiment}{%
    \item Rilevazione del sentimento di un testo
}{%
    \item \texttt{analysis/language}
}

\crc{analysis/stopwords}{%
    \item Rimozione di stop words da un testo
}{%
    \item \texttt{analysis/language}
}


\subsubsection*{Scacchi}

\crc{tweet/twitter\_oauth}{%
    \item Autenticazione a Twitter
}{%
}

\crc{tweet/send}{%
    \item Pubblicazione di un tweet
}{%
    \item \texttt{tweet/twitter\_oauth}
}

\crc{ChessOpponent}{%
    \item Pubblicazione dell'immagine con lo stato della scacchiera
    \item Selezione della mossa successiva a maggioranza
}{%
    \item \texttt{tweet/send}
}

\crc{ChessGame}{%
    \item Tracciamento dello stato della partita
    \item Rilevamento conclusione partita
}{%
}

\subsubsection*{Reazione a catena e L'Eredità}
\crc{games/winningWord}{%
    \item Estrazione della parola vincente
}{%
    \item \texttt{fetch/keyword}
}
\crc{games/userAttempts}{%
    \item Raccolta dei tentativi degli utenti
}{%
    \item \texttt{fetch/keyword}
}

\subsubsection*{Fantacitorio}
\crc{games/fantacitorio}{%
    \item Raccolta dei punteggi settimanali
    \item Calcolo della classifica generale
    \item Raccolta delle squadre degli utenti
    \item Calcolo di statistiche sulla classifica
}{%
    \item \texttt{fetch/keyword}
    \item \texttt{fetch/user}
}